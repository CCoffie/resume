%%%%%%%%%%%%%%%%%%%%%%%%%%%%%%%%%%%%%%%%%
% Caleb Coffie's Resume/CV
% XeLaTeX Template
% Version 2.0 (1/22/15)
%
% Original author:
% Adrien Friggeri (adrien@friggeri.net)
% https://github.com/afriggeri/CV
%
% Important notes:
% This template needs to be compiled with XeLaTeX.
%
%%%%%%%%%%%%%%%%%%%%%%%%%%%%%%%%%%%%%%%%%

\documentclass[]{CalebCoffie-CV-Class} % Add 'print' as an option into the square bracket to remove colors from this template for printing

\begin{document}

\header{Caleb}{Coffie}{Application Security Engineer} % Your name and current job title/field

%----------------------------------------------------------------------------------------
%	SIDEBAR SECTION
%----------------------------------------------------------------------------------------

\begin{aside} % In the aside, each new line forces a line break
\section{Contact}
1416 Weatherford Dr.
Austin, TX 78753
~
+1 (518) 982-6860
~
\href{mailto:CalebCoffie@gmail.com}{CalebCoffie@gmail.com}
\href{https://CalebCoffie.com}{https://CalebCoffie.com}
\href{https://github.com/CCoffie}{github://CCoffie}
\section{Skills}
\item Python {\color{black} $\varheartsuit$}
\item Django
\item React.js \& Redux
\item Docker
\item git
\section{Hobbies}
\item Coffee
\item Smoking Meat
\item Home Automation
\item Hardware Hacking
\end{aside}

%----------------------------------------------------------------------------------------
%	WORK EXPERIENCE SECTION
%----------------------------------------------------------------------------------------

\section{Experience}

\begin{entrylist}
%------------------------------------------------
\entry
{June 2016--\\Now}
{Indeed}
{Austin, Texas}
{\emph{Application Security Engineer} \\
Designed and built an autonomous web application security pipeline. This system regularly would find vulnerabilities before our public bug bounty program. It consisted of 10 different custom built services that performed things such static code analysis, dynamic web vulnerability scanning, and application discovery. \\

Participated Indeed University, a fast pace development program, in which teams of 4 created their own beta Indeed products in 3 months. On this team I was in charge of our marketing campaigns as well as our frontend development.}
%------------------------------------------------
\entry
{May 2015--\\August 2015}
{Indeed}
{Austin, Texas}
{\emph{Security Intern} \\
Created a system around the Nessus vulnerability scanner. This system created a workflow around the Nessus results, allowing for quick tracking of the vulnerabilities. It also placed the findings into the centralized logging platform ElasticSearch which allowed for in depth analysis of the Nessus results.}
%------------------------------------------------
\end{entrylist}

%----------------------------------------------------------------------------------------
%	EDUCATION SECTION
%----------------------------------------------------------------------------------------

\section{Education}

\begin{entrylist}
%------------------------------------------------
\entry
{May 2016}
{B.S. {\normalfont Computing Security}}
{Rochester Institute of Technology}
{Minor in Open Source \& Free Culture}
%------------------------------------------------
\end{entrylist}

%----------------------------------------------------------------------------------------
%	CONFERENCES SECTION
%----------------------------------------------------------------------------------------
\section{Conferences}

\textbf{LASCON} 2017 Presented on \href{http://sched.co/C1HZ}{Improving dynamic vulnerability scanners with static code analysis}

%----------------------------------------------------------------------------------------


%----------------------------------------------------------------------------------------
%	PROJECTS SECTION
%----------------------------------------------------------------------------------------

\section{Projects}

\begin{entrylist}
%------------------------------------------------
\entry
{Summer 2017}
{\href{https://github.com/indeedsecurity/wes/}{WES}}
{}
{Designed and built a source code analysis tool in Python. This tool is used to parse Java Spring, Java Servlet, and Python Django web applications to find HTTP endpoints, parameters, and much more. The main use of this project was to use the data returned to run a dynamic web vulnerability scanner without the need for crawling while providing better code coverage. The data can also be useful for investigating web application security vulnerabilities since it provides a direct mapping of url to line of code within the project.}
%------------------------------------------------
\entry
{Winter 2015}
{\href{https://github.com/Roastero/Openroast/}{OpenRoast}}
{}
{Reverse engineered a USB controlled coffee roaster. Wrote a custom, cross-platform application in Python and PyQt which significantly extends the capabilities of the roaster, providing a thermostat-based controller and roast graphs.}
%------------------------------------------------
\end{entrylist}
\end{document}