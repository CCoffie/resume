%%%%%%%%%%%%%%%%%%%%%%%%%%%%%%%%%%%%%%%%%
% Caleb Coffie's Resume/CV
% XeLaTeX Template
% Version 2.0 (1/22/15)
%
% Original author:
% Adrien Friggeri (adrien@friggeri.net)
% https://github.com/afriggeri/CV
%
% Important notes:
% This template needs to be compiled with XeLaTeX.
%
%%%%%%%%%%%%%%%%%%%%%%%%%%%%%%%%%%%%%%%%%

\documentclass[]{CalebCoffie-CV-Class} % Add 'print' as an option into the square bracket to remove colors from this template for printing

\begin{document}

\header{Caleb}{Coffie}{Computer Security Buff \& Part-time Developer} % Your name and current job title/field

%----------------------------------------------------------------------------------------
%	SIDEBAR SECTION
%----------------------------------------------------------------------------------------

\begin{aside} % In the aside, each new line forces a line break
\section{Contact}
XXXX XXXXXXXXXX XXX
Schenectady, NY 12303
~
+X (XXX) XXX XXX
~
\href{mailto:CalebCoffie@gmail.com}{CalebCoffie@gmail.com}
\href{https://CalebCoffie.com}{https://CalebCoffie.com}
\href{https://github.com/johnsmith}{github://CCoffie}
\section{Skills}
Python {\color{black} $\varheartsuit$}, Flask, git, Nessus, VMware Workstation \& VSphere, Clonezilla (DRBL)
\section{Relevant Courses}
\item Cryptography and Authentication
\item Cyber Defense Techniques
\item Humanitarian Free/Open Source Software
\item Free \& Open Source Culture
\item Information Security Policies
\item Risk Management for Information Security
\item Computer System Forensics
\item Windows System Forensics
\end{aside}

%----------------------------------------------------------------------------------------
%	EDUCATION SECTION
%----------------------------------------------------------------------------------------

\section{Education}

\begin{entrylist}
%------------------------------------------------
\entry
{Expected Grad.\\May 2016}
{B.S. {\normalfont Computing Security}}
{Rochester Institute of Technology - Rochester, NY}
{Minor in Open Source \& Free Culture}
%------------------------------------------------
\end{entrylist}

%----------------------------------------------------------------------------------------
%	WORK EXPERIENCE SECTION
%----------------------------------------------------------------------------------------

\section{Experience}

\begin{entrylist}
%------------------------------------------------
\entry
{May 2015--\\August 2015}
{Indeed}
{Austin, Texas}
{\emph{Security Intern} \\
Created a system around the Nessus vulnerability scanner. This system created a workflow around the Nessus results, allowing for quick tracking of the vulnerabilities. It also placed the findings into the centralized logging platform ElasticSearch which allowed for in depth analysis of the Nessus results.}

%------------------------------------------------
\entry
{January 2014--\\August 2014}
{Wyman-Gordon Forging}
{Houston, Texas}
{\emph{IT/Networking Systems Administrator} \\
Supported over 500 users on Windows systems. Worked with users to troubleshoot issues with hardware, software and networking issues. Led the effort of transitioning 200+ Windows XP user and control systems to newer operating systems. Developed and maintained an imaging server that used to setup systems multiple plants around the country. Implemented various asset and device management systems.}
%------------------------------------------------
\entry
{September 2012--\\Now}
{RIT Center of Imaging Science}
{Rochester, NY}
{\emph{Student Systems Administrator} \\
Provided support for Student and Faculty using Mac OS X, Windows, and Linux systems. Imaged and deployed systems for new classes.}
%------------------------------------------------
\end{entrylist}

%----------------------------------------------------------------------------------------
%	PROJECTS SECTION
%----------------------------------------------------------------------------------------

\section{Projects}

\begin{entrylist}
%------------------------------------------------
\entry
{Winter 2015}
{OpenRoast}
{}
{Reverse engineered a USB controlled coffee roaster. Wrote a custom, cross-platform application in Python and PyQt which significantly extends the capabilities of the roaster, providing a thermostat-based controller and roast graphs.}
%------------------------------------------------
\entry
{Spring 2014}
{Kippo-Graph}
{Open Source Project Contribution}
{Added a feature to the project called Kippo-PlayLog. This new featured allowed for an animated playback of the attacker’s SSH log within browser.}
%------------------------------------------------
\end{entrylist}

%----------------------------------------------------------------------------------------
%	ACTIVITIES SECTION
%----------------------------------------------------------------------------------------
\section{Activities}

\textbf{Clubs:} Security Practices and Research Asssociation (SPARSA) \textbf{Competitions:} Cyber Security Awareness Week (CSAW), Information Security Talent Search (ISTS), National Cyber League (NCL), Various Hackathons (HackRPI, PennApps, Hack The Planet, and More)

%----------------------------------------------------------------------------------------

%----------------------------------------------------------------------------------------
%	INTERESTS AND ACTIVITIES SECTION
%----------------------------------------------------------------------------------------

\section{Hobbies}

Church, Coffee, Hardware Hacking, Computer System Building, Quadcopters
%----------------------------------------------------------------------------------------

\end{document}